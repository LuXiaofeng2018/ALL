\documentclass[12pt]{article}
\usepackage{amsmath}
\usepackage{ amssymb }
\usepackage{breqn}
\begin{document}

\title{SWW soliton}
\author{Jordan Pitt / u5013521}

\section{SWW soliton}
The continuity equation is:
\begin{equation}
\label{coneq1}
h_t + uh_x + u_xh = 0
\end{equation}

the momentum equation is
\[\left(uh\right)_t + \left(u^2h + g\frac{h^2}{2}\right)_x = 0\]
expanding

\[u_th + uh_t + \left(u^2h\right)_x + ghh_x = 0\]

\[u_th + uh_t + 2uu_xh + u^2h_x  + ghh_x = 0\]

using \eqref{coneq1}

\[u_th - u^2h_x - uu_xh  + 2uu_xh + u^2h_x  + ghh_x = 0\]

\[u_th + uu_xh  + ghh_x = 0\]
when $h > 0$
\begin{equation}
\label{veleq1}
u_t + uu_x  + gh_x = 0
\end{equation}

want solutions to \eqref{coneq1} and \eqref{veleq1} such that

\[h(x,t) = \phi(x - ct)\]
\[u(x,t) = \psi(x - ct)\]

Substituting these into \eqref{coneq1} gives

\[-c\phi'(x - ct) + \psi(x - ct)\phi'(x - ct) + \psi'(x - ct)\phi(x - ct) = 0\]

\[\left(\psi(x - ct) -c\right)\phi'(x - ct) + \psi'(x - ct)\phi(x - ct) = 0\]

\begin{equation}
\label{expconeq}
\left(\psi(x - ct) -c\right)\phi'(x - ct) + \psi'(x - ct)\phi(x - ct) = 0
\end{equation}

Substituting these into \eqref{veleq1} gives
\[-c\psi'(x - ct) + \psi(x - ct)\psi'(x - ct)  + g\phi'(x-ct) = 0\]
\[\left(\psi(x - ct) -c\right)\psi'(x - ct) + g\phi'(x-ct) = 0\]

\begin{equation}
\label{expveleq}
\left(\psi(x - ct) -c\right)\psi'(x - ct) + g\phi'(x-ct) = 0
\end{equation}

So we find that by \eqref{expveleq}
\[\left(\psi(x - ct) -c\right) = \frac{-g\phi'(x-ct)}{\psi'(x - ct)} \]

subbing this into \eqref{expconeq}
\[\frac{-g\phi'(x-ct)}{\psi'(x - ct)}\phi'(x - ct) + \psi'(x - ct)\phi(x - ct) = 0\]

\[g \left(\phi'(x - ct)\right)^2 = \left(\psi'(x - ct)\right)^2\phi(x - ct)\]
\[g \frac{\left(\phi'(x - ct)\right)^2}{\phi(x - ct)} = \left(\psi'(x - ct)\right)^2\]

\[\sqrt{g} \frac{\phi'(x - ct)}{\sqrt{\phi(x - ct)}} = \psi'(x - ct)\]

After we differentiated with respect to x and t it will be easier to see what we have with a change of variables so $y = x - ct$:

\[\sqrt{g} \frac{\phi'(y)}{\sqrt{\phi(y)}} = \psi'(y)\]

and 

\begin{equation}
\label{expconeqy}
\left(\psi(y) -c\right)\phi'(y) + \psi'(y)\phi(y) = 0
\end{equation}
\begin{equation}
\label{expveleqy}
\left(\psi(y) -c\right)\psi'(y) + g\phi'(y) = 0
\end{equation}

We know by the fundamental theorem of calculus that
\[\psi(y) = \int_0^y \psi'(s) ds - \psi(0)\]

\[\psi(y) = \sqrt{g} \int_0^y \frac{\phi'(s)}{\sqrt{\phi(s)}} ds + \psi(0)\]

It is quite easy to see that the antiderivative of $\frac{\phi'(s)}{\sqrt{\phi(s)}}$  is $2\sqrt{\phi(s)}$

\[\psi(y) = 2\sqrt{g} \left[\sqrt{\phi(y)} - \sqrt{\phi(0)}\right] + \psi(0)\]

\[\psi(y) = 2\sqrt{g}\sqrt{\phi(y)} - 2\sqrt{g}\sqrt{\phi(0)} + \psi(0)\]

defining $d =- 2\sqrt{g}\sqrt{\phi(0)} + \psi(0)$ which is a constant

\[\psi(y) = 2\sqrt{g}\sqrt{\phi(y)} + d\]

So we have that (assume $c \neq 0$)
\begin{equation}
\label{relation}
u(x,t) = 2\sqrt{g}\sqrt{h(x,t)} + d  
\end{equation}










\end{document}
