\documentclass[12pt]{article}
\usepackage{amsmath}
\usepackage{ amssymb }
\usepackage{breqn}
\begin{document}

\section{Elliptic Equation}
The linearised elliptic equation is

\[G = Hu - \frac{H^3}{3}u_{xx}\]

Want to find out the FEM approximation to this $G'$ such that

\[G' = \mathcal{G}_{FE_1} u\]

for $P^1$ FEM

\[G' = \mathcal{G}_{FE_2} u\]

for $P^2$ FEM.

To do so we begin by first multiplying by an arbitrary test function $v$ so that

\[Gv = Huv - \frac{H^3}{3}u_{xx}v\]

and then we integrate over the entire domain to get 
\[\int_\Omega Gv dx = \int_\Omega Huv dx - \int_\Omega \frac{H^3}{3}u_{xx}vdx\]

for all $v$

We then make use of integration by parts, with Dirchlet boundaries to get

\[\int_\Omega Gv dx = \int_\Omega Huv dx + \int_\Omega \frac{H^3}{3}u_{x}v_xdx\]

Our FVM discretisation already has a natrual structure with linear functions intervals of $[x_{j- 1/2} , x_{j+1/2}]$, to achieve this in $P^1$ we have our nodes at the boundaries, thus

So we can reformulate this as 

\[\sum_{j}\int_{x_{j-1/2}}^{x_{j+3/2}} Gv dx = \sum_{j}\int_{x_{j-1/2}}^{x_{j+3/2}} Huv dx + \sum_{j}\int_{x_{j-1/2}}^{x_{j+3/2}} \frac{H^3}{3}u_{x}v_{x}dx\]

or more aptly

\[\sum_{j}\int_{x_{j-1/2}}^{x_{j+3/2}} Gv dx - \int_{x_{j-1/2}}^{x_{j+3/2}} Huv dx - \int_{x_{j-1/2}}^{x_{j+3/2}} \frac{H^3}{3}u_{x}v_{x}dx = 0 \]

for all v

\[\sum_{j}\int_{x_{j-1/2}}^{x_{j+3/2}} Gv dx - H\int_{x_{j-1/2}}^{x_{j+3/2}} uv dx -  \frac{H^3}{3}\int_{x_{j-1/2}}^{x_{j+3/2}}u_{x}v_{x}dx = 0 \]

\section{P1 FEM}
We are going to corodainte tranform from x space the interval $[x_{i-1/2},x_{i+1/2} ,x_{i+3/2}]$ to the $\xi$ space interval $[-1,0,1]$. To accomplish this we have the following relation

$$x = \xi\Delta x + x_{j+1/2}$$

Taking the derivative we see

$dx = d\xi\Delta x$ , $\frac{dx}{d\xi} = \Delta x$ , $\frac{d\xi}{dx} = \frac{1}{\Delta x}$ 


For this FEM we are intereseted in $G_{i+1/2}$ and then we can just get a shift operator to get the otherones. For FEM we replace the functions by their P1 approximations so

\[G \approx G' = \sum_{j}G_{j+ 1/2}v_{j+ 1/2}\]
\[u \approx u' = \sum_{j}u_{j+ 1/2}v_{j+ 1/2}\]

\[\sum_{j}\int_{x_{j-1/2}}^{x_{j+3/2}} G'v dx - H\int_{x_{j-1/2}}^{x_{j+3/2}} u'v dx -  \frac{H^3}{3}\int_{x_{j-1/2}}^{x_{j+3/2}}u'_{x}v_{x}dx = 0 \]

We break this up into the integrals because of the domain of dependence of the basis functions is covered. We also just use a particular basis function as the test function, in particular we choose $v_{j + 1/2}$

\[\int_{x_{j-1/2}}^{x_{j+3/2}} G'(x)v_{j + 1/2} dx = \int_{-1}^{1} G'(\xi)v_{j + 1/2}(\xi) \frac{d x}{d\xi}d\xi\]

\[= \Delta x \int_{-1}^{1} \left(G_{j- 1/2}v_{j - 1/2} + G_{j+ 1/2}v_{j+ 1/2} +G_{j+ 3/2}v_{j+ 3/2} \right)v_{j + 1/2} d\xi\]

\[= \Delta x \left[G_{j- 1/2} \int_{-1}^{1} v_{j - 1/2}v_{j + 1/2} d\xi + G_{j+ 1/2} \int_{-1}^{1} v_{j + 1/2}v_{j + 1/2} d\xi +  G_{j+3/2} \int_{-1}^{1} v_{j + 3/2}v_{j + 1/2} d\xi\right]\]

We have the 

\[\int_{-1}^{1} v_{j - 1/2}v_{j + 1/2} d\xi =  \int_{-1}^{1} v_{j + 3/2}v_{j + 1/2} d\xi = \frac{1}{6}\]

\[\int_{-1}^{1} v_{j + 1/2}v_{j + 1/2} d\xi = \frac{2}{3}\]

\[= \Delta x \left[G_{j- 1/2} \frac{1}{6} + G_{j+ 1/2} \frac{2}{3} +  G_{j+3/2} \frac{1}{6}\right]\]

\[=  \frac{\Delta x}{6} \left[G_{j- 1/2} + 4G_{j+ 1/2} +  G_{j+3/2} \right]\]

Similarly we have

\[- H\int_{x_{j-1/2}}^{x_{j+3/2}} u'v_{j+ 1/2} dx = -\frac{H\Delta x}{6} \left[u_{j- 1/2} + 4u_{j+ 1/2} +  u_{j+3/2} \right] \]



\[-\frac{H^3}{3}\int_{x_{j-1/2}}^{x_{j+3/2}}u'_{x}(v_{j+ 1/2})_{x}dx = -\frac{H^3}{3}\int_{-1}^{1}u'_{\xi}\frac{d \xi }{dx}(v_{j+ 1/2})_{\xi}\frac{d \xi }{dx} \frac{d x}{d\xi} d\xi   \]

\[= -\frac{H^3}{3\Delta x}\int_{-1}^{1}u'_{\xi}(v_{j+ 1/2})_{\xi} d\xi   \]

where $'$ denotes derivative
\[= -\frac{H^3}{3\Delta x}\int_{-1}^{1}\left(u_{j- 1/2}'v_{j - 1/2}' + u_{j+ 1/2}'v_{j+ 1/2}' +u_{j+ 3/2}'v_{j+ 3/2}' \right)v_{j+ 1/2}' d\xi   \]

\[= -\frac{H^3}{3\Delta x}\left[u_{j- 1/2} \int_{-1}^{1} v_{j - 1/2}'v_{j + 1/2}' d\xi + u_{j+ 1/2} \int_{-1}^{1} v_{j + 1/2}'v_{j + 1/2}' d\xi +  u_{j+3/2} \int_{-1}^{1} v_{j + 3/2}'v_{j + 1/2}' d\xi\right]   \]

We have that 

\[\int_{-1}^{1} v_{j - 1/2}'v_{j + 1/2}' d\xi = -1 = \int_{-1}^{1} v_{j + 3/2}'v_{j + 1/2}' d\xi \]

\[\int_{-1}^{1} v_{j + 1/2}'v_{j + 1/2}' d\xi = 2\]

\[= -\frac{H^3}{3\Delta x}\left[-u_{j- 1/2}  + 2u_{j+ 1/2} -  u_{j+3/2} \right]   \]

Then our equation becomes

\begin{multline}
\frac{\Delta x}{6} \left[G_{j- 1/2} + 4G_{j+ 1/2} +  G_{j+3/2} \right] = \\ \frac{H\Delta x}{6} \left[u_{j- 1/2} + 4u_{j+ 1/2} +  u_{j+3/2} \right] +  \frac{H^3}{3\Delta x}\left[-u_{j- 1/2}  + 2u_{j+ 1/2} -  u_{j+3/2} \right]
\end{multline}

\begin{multline}
\left[G_{j- 1/2} + 4G_{j+ 1/2} +  G_{j+3/2} \right] = \\ H \left[u_{j- 1/2} + 4u_{j+ 1/2} +  u_{j+3/2} \right] +  \frac{2H^3}{\Delta x^2}\left[-u_{j- 1/2}  + 2u_{j+ 1/2} -  u_{j+3/2} \right]
\end{multline}

This formula is correct for $u= 1,x,x^2$

By shifting we also get

\begin{multline}
\left[G_{j- 3/2} + 4G_{j-1/2} +  G_{j+1/2} \right] = \\ H \left[u_{j- 3/2} + 4u_{j- 1/2} +  u_{j+1/2} \right] +  \frac{2H^3}{\Delta x^2}\left[-u_{j- 3/2}  + 2u_{j- 1/2} -  u_{j+1/2} \right]
\end{multline}

\begin{multline}
\left[G_{j+ 1/2} + 4G_{j+ 3/2} +  G_{j+5/2} \right] = \\ H \left[u_{j+1/2} + 4u_{j+ 3/2} +  u_{j+5/2} \right] +  \frac{2H^3}{\Delta x^2}\left[-u_{j+ 1/2}  + 2u_{j+ 3/2} -  u_{j+5/2} \right]
\end{multline}

We begin by assuming the analytic structure for $G$ and $u$ (to get easy shift operators).
Let quantity $q$ is given by so that
$q(x,t) = q(0,0) e^{i\left(\omega t + kx\right)}$. The use of this comes when we use our uniform grid in space, so that $q(x_j,t) = q_j$ then $q_{j \pm l} = q_j e^{\pm ik l\Delta x} $

Then we have 

\begin{multline}
\left[G_j e^{- ik \frac{1}{2}\Delta x} + 4G_j e^{ik \frac{1}{2}\Delta x}  +  G_j e^{ik \frac{3}{2}\Delta x}  \right] = \\ H \left[u_j e^{- ik \frac{1}{2}\Delta x} + 4u_j e^{ik \frac{1}{2}\Delta x}  +  u_j e^{ik \frac{3}{2}\Delta x}  \right] +  \frac{2H^3}{\Delta x^2}\left[-u_j e^{- ik \frac{1}{2}\Delta x}  + 2u_j e^{ik \frac{1}{2}\Delta x} -  u_j e^{ik \frac{3}{2}\Delta x} \right]
\end{multline}

\begin{multline}
G_j\left[e^{- ik \frac{1}{2}\Delta x} + 4 e^{ik \frac{1}{2}\Delta x}  +  e^{ik \frac{3}{2}\Delta x}  \right] = \\ \left( H \left[e^{- ik \frac{1}{2}\Delta x} + 4 e^{ik \frac{1}{2}\Delta x}  +  e^{ik \frac{3}{2}\Delta x}  \right] +  \frac{2H^3}{\Delta x^2}\left[- e^{- ik \frac{1}{2}\Delta x}  + 2 e^{ik \frac{1}{2}\Delta x} -  e^{ik \frac{3}{2}\Delta x} \right] \right)u_j
\end{multline}

\begin{multline}
G_j= \\ \left( H  +  \frac{2H^3}{\Delta x^2}\frac{\left[- e^{- ik \frac{1}{2}\Delta x}  + 2 e^{ik \frac{1}{2}\Delta x} -  e^{ik \frac{3}{2}\Delta x} \right]}{\left[e^{- ik \frac{1}{2}\Delta x} + 4 e^{ik \frac{1}{2}\Delta x}  +  e^{ik \frac{3}{2}\Delta x}  \right]} \right)u_j
\end{multline}

\begin{multline}
G_j= \\ \left( H  +  \frac{2H^3}{\Delta x^2}\frac{\left[2i\sin\left(k \frac{1}{2}\Delta x\right) +  e^{ik \frac{1}{2}\Delta x} -  e^{ik \frac{3}{2}\Delta x} \right]}{\left[e^{- ik \frac{1}{2}\Delta x} + 4 e^{ik \frac{1}{2}\Delta x}  +  e^{ik \frac{3}{2}\Delta x}  \right]} \right)u_j
\end{multline}

\begin{multline}
G_j= \\ \left( H  +  \frac{2H^3}{\Delta x^2}\frac{2i\sin\left(k \frac{1}{2}\Delta x\right) +  e^{ik\Delta x}\left(e^{-ik \frac{1}{2}\Delta x} -  e^{ik \frac{1}{2}\Delta x} \right) }{\left[e^{- ik \frac{1}{2}\Delta x} + 4 e^{ik \frac{1}{2}\Delta x}  +  e^{ik \frac{3}{2}\Delta x}  \right]} \right)u_j
\end{multline}

\begin{multline}
G_j= \\ \left( H  +  \frac{2H^3}{\Delta x^2}\frac{2i\sin\left(k \frac{1}{2}\Delta x\right) -  2ie^{ik\Delta x}\sin\left(k \frac{1}{2}\Delta x\right) }{\left[e^{- ik \frac{1}{2}\Delta x} + 4 e^{ik \frac{1}{2}\Delta x}  +  e^{ik \frac{3}{2}\Delta x}  \right]} \right)u_j
\end{multline}

\begin{multline}
G_j= \\ \left( H  +  \frac{2H^3}{\Delta x^2}\frac{2i\sin\left(k \frac{1}{2}\Delta x\right) -  2ie^{ik\Delta x}\sin\left(k \frac{1}{2}\Delta x\right) }{\left[ 2\cos\left(k \frac{1}{2}\Delta x\right)+ 2 e^{ik \frac{1}{2}\Delta x} + e^{ik \frac{1}{2}\Delta x}  +  e^{ik \frac{3}{2}\Delta x}  \right]} \right)u_j
\end{multline}

\begin{multline}
G_j= \\ \left( H  +  \frac{2H^3}{\Delta x^2}\frac{2i\sin\left(k \frac{1}{2}\Delta x\right) -  2ie^{ik\Delta x}\sin\left(k \frac{1}{2}\Delta x\right) }{\left[ 2\cos\left(k \frac{1}{2}\Delta x\right)+ 2 e^{ik \frac{1}{2}\Delta x} + 2e^{ik\Delta x}\cos\left(k \frac{1}{2}\Delta x\right)  \right]} \right)u_j
\end{multline}

\begin{multline}
G_j= \\ \left( H  +  \frac{2H^3 i}{\Delta x^2}\frac{\sin\left(k \frac{1}{2}\Delta x\right) -  e^{ik\Delta x}\sin\left(k \frac{1}{2}\Delta x\right) }{\left[\cos\left(k \frac{1}{2}\Delta x\right)+ e^{ik \frac{1}{2}\Delta x} + e^{ik\Delta x}\cos\left(k \frac{1}{2}\Delta x\right)  \right]} \right)u_j
\end{multline}

\begin{multline}
G_j= \\ \left( H  +  \frac{2H^3 i}{\Delta x^2}\frac{\left(1 -e^{ik\Delta x} \right)\sin\left(k \frac{1}{2}\Delta x\right) }{\left(1  + e^{ik\Delta x} \right)\cos\left(k \frac{1}{2}\Delta x\right)+ e^{ik \frac{1}{2}\Delta x} } \right)u_j
\end{multline}

So we have

\[\mathcal{G}_{FEM_2} = \left( H  +  \frac{2H^3 i}{\Delta x^2}\frac{\left(1 -e^{ik\Delta x} \right)\sin\left(k \frac{1}{2}\Delta x\right) }{\left(1  + e^{ik\Delta x} \right)\cos\left(k \frac{1}{2}\Delta x\right)+ e^{ik \frac{1}{2}\Delta x} } \right) \]

With taylor expansion

\[\mathcal{G}_{FEM_2} = H + \frac{H^3k^2}{3} + O(\Delta x ^2)\]

as desired.


\end{document}
