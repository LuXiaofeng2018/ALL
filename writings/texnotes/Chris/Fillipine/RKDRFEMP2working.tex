\documentclass[12pt]{article}
\usepackage{amsmath}
\usepackage{ amssymb }
\usepackage{breqn}
\begin{document}

\section{Elliptic Equation}
The linearised elliptic equation is

\[G = Hu - \frac{H^3}{3}u_{xx}\]

Want to find out the FEM approximation to this $G'$ such that

\[G' = \mathcal{G}_{FE_1} u\]

for $P^1$ FEM

\[G' = \mathcal{G}_{FE_2} u\]

for $P^2$ FEM.

To do so we begin by first multiplying by an arbitrary test function $v$ so that

\[Gv = Huv - \frac{H^3}{3}u_{xx}v\]

and then we integrate over the entire domain to get 
\[\int_\Omega Gv dx = \int_\Omega Huv dx - \int_\Omega \frac{H^3}{3}u_{xx}vdx\]

for all $v$

We then make use of integration by parts, with Dirchlet boundaries to get

\[\int_\Omega Gv dx = \int_\Omega Huv dx + \int_\Omega \frac{H^3}{3}u_{x}v_xdx\]

Our FVM discretisation already has a natrual structure with linear functions intervals of like $[x_{j- 1/2} , x_{j+1/2}]$

So we can reformulate this as 

\[\sum_{j}\int_{x_{j-1/2}}^{x_{j+1/2}} Gv dx = \sum_{j}\int_{x_{j-1/2}}^{x_{j+1/2}} Huv dx + \sum_{j}\int_{x_{j-1/2}}^{x_{j+1/2}} \frac{H^3}{3}u_{xx}vdx\]

or more aptly

\[\sum_{j}\int_{x_{j-1/2}}^{x_{j+1/2}} Gv dx - \int_{x_{j-1/2}}^{x_{j+1/2}} Huv dx - \int_{x_{j-1/2}}^{x_{j+1/2}} \frac{H^3}{3}u_{xx}vdx = 0 \]

for all v

\[\sum_{j}\int_{x_{j-1/2}}^{x_{j+1/2}} Gv dx - H\int_{x_{j-1/2}}^{x_{j+1/2}} uv dx -  \frac{H^3}{3}\int_{x_{j-1/2}}^{x_{j+1/2}}u_{xx}vdx = 0 \]

Each of these integrals has been computed by Chris previously, for the basis functions that are non-zero on the element. We include just the basis function which is 1 at $x_j$

\[\int_{x_{j-1/2}}^{x_{j+1/2}} Gv dx = \frac{dx}{d\xi}\int_{-1}^{1} Gv d\xi  = \frac{\Delta x}{15} \left[
G^+_{j - 1/2} + 8G_{j} + G^{-}_{j+1/2} \right]\]


\[H\int_{x_{j-1/2}}^{x_{j+1/2}} uv dx = \frac{dx}{d\xi}H\int_{-1}^{1} uv d\xi =  \frac{H\Delta x}{15} \left[
u_{j - 1/2} + 8u_{j} + u_{j+1/2} \right]\]

\[\frac{H^3}{3}\int_{x_{j-1/2}}^{x_{j+1/2}}u_{x}v_{x}dx = \frac{d\xi}{dx}\frac{H^3}{3}\int_{-1}^{1} u_\xi v_\xi d\xi = \frac{H^3}{3} \frac{2}{3\Delta x}\left[ -
4u_{j - 1/2} + 8u_{j} - 4u_{j+1/2} \right]\]

So we ahve 

\begin{multline}
 \frac{\Delta x}{15} \left[
 G^+_{j - 1/2} + 8G_{j} + G^{-}_{j+1/2} \right] \\= \frac{H\Delta x}{15} \left[
 u_{j - 1/2} + 8u_{j} + u_{j+1/2} \right] + \frac{H^3}{3} \frac{2}{3\Delta x}\left[ -
 4u_{j - 1/2} + 8u_{j} - 4u_{j+1/2} \right]
\end{multline}

\begin{multline}
\frac{\Delta x}{15} \left[
G^+_{j - 1/2} + 8G_{j} + G^{-}_{j+1/2} \right] \\= \frac{H\Delta x}{15} \left[
u_{j - 1/2} + 8u_{j} + u_{j+1/2} \right] + \frac{H^3}{3} \frac{8}{3\Delta x}\left[ -
u_{j - 1/2} + 2u_{j} - u_{j+1/2} \right]
\end{multline}

\begin{multline}
\frac{1}{5} \left[
G^+_{j - 1/2} + 8G_{j} + G^{-}_{j+1/2} \right] \\= \frac{H\Delta x}{5} \left[
u_{j - 1/2} + 8u_{j} + u_{j+1/2} \right] + \frac{H^3}{3} \frac{8}{\Delta x^2}\left[ -
u_{j - 1/2} + 2u_{j} - u_{j+1/2} \right]
\end{multline}

\begin{multline}
 \left[
G^+_{j - 1/2} + 8G_{j} + G^{-}_{j+1/2} \right] \\= H \left[
u_{j - 1/2} + 8u_{j} + u_{j+1/2} \right] + \frac{H^3}{3} \frac{40}{\Delta x^2}\left[ -
u_{j - 1/2} + 2u_{j} - u_{j+1/2} \right]
\end{multline}

This formula works, then using our shift operators we can do

\begin{multline}
\left[
e^{-ik\Delta x}\mathcal{R}^+ + 8 + \mathcal{R}^- \right] G_{j} \\=\left( H \left[
e^{-ik\Delta x/2} + 8 + e^{ik\Delta x/2}\right] + \frac{H^3}{3} \frac{40}{\Delta x^2}\left[ -
e^{-ik\Delta x/2} + 2 - e^{ik\Delta x/2} \right] \right) u_j
\end{multline}

\begin{multline}
 G_{j} \\= \frac{1}{
 	e^{-ik\Delta x}\mathcal{R}^+ + 8 + \mathcal{R}^-}\left( H \left[8 + 2 \cos\left(k\Delta x/2\right) \right] + \frac{H^3}{3} \frac{40}{\Delta x^2}\left[ 2 - 2 \cos\left(k\Delta x/2\right) \right] \right) u_j
\end{multline}

\begin{multline}
G_{j} \\= \frac{2}{
	e^{-ik\Delta x}\mathcal{R}^+ + 8 + \mathcal{R}^-}\left( H \left[4 +  \cos\left(k\Delta x/2\right) \right] + \frac{H^3}{3} \frac{40}{\Delta x^2}\left[1 -  \cos\left(k\Delta x/2\right) \right] \right) u_j
\end{multline}













\end{document}
