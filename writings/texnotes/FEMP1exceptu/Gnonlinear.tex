\documentclass[12pt]{article}
\usepackage{amsmath}
\usepackage{ amssymb }
\usepackage{breqn}
\begin{document}

\section{Elliptic Equation}
The linearised elliptic equation is
\[G = uh - \frac{\partial}{\partial x}\left(\frac{h^3}{3}u_x\right)\]

\[G = uh - h^2h_xu_x - \frac{h^3}{3}u_{xx}\]

now we replace

\[u = U(t)e^{ikx}\]
\[h = H(t)e^{ikx}\]

\[G = UH e^{2ikx} - H^2 e^{2ikx} ikH e^{ikx} ikU e^{ikx}  - \frac{1}{3} H^3 e^{3ikx} (-k^2 )U e^{ikx}\]

\[G = UH e^{2ikx} +k^2 H^3U e^{4ikx}  +  \frac{k^2}{3} UH^3 e^{4ikx}\]

\[G = \left(1 + \frac{4}{3}k^2 H^2 e^{2ikx} \right)UH e^{2ikx}\]
\[G_j = \left(1 + \frac{4}{3}k^2 H^2 e^{2ikx_j} \right)UH e^{2ikx_j}\]

\section{Finite Difference}

we have the derivatives

\[\left(\frac{\partial q}{\partial x}\right)_j= \frac{q_{j+1} - q_{j-1}}{2\Delta x} = \frac{e^{ik\Delta x} - e^{-ik\Delta x}}{2\Delta x} Q(t)e^{ikx_j} =\frac{i\sin\left(k\Delta x\right)}{\Delta x} Q(t)e^{ikx_j} \]

\[\left(\frac{\partial^2 q}{\partial x^2}\right)_j= \frac{q_{j+1} -2q_j + q_{j-1}}{\Delta x^2} = \frac{e^{ik\Delta x} - 2 + e^{-ik\Delta x}}{\Delta x^2} Q(t)e^{ikx_j} =\frac{2\cos\left(k\Delta x\right) - 2}{\Delta x^2} Q(t)e^{ikx_j} \]
\[\left(\frac{\partial^2 q}{\partial x^2}\right)_j=  -4\frac{\sin^2\left(\frac{k\Delta x}{2}\right)}{\Delta x^2} Q(t)e^{ikx_j} =  - \left(\frac{2\sin\left(\frac{k\Delta x}{2}\right)}{\Delta x}\right)^2 Q(t)e^{ikx_j}\]

\[G = uh - h^2h_xu_x - \frac{h^3}{3}u_{xx}\]

\[G_j = UH e^{2ikx_j} - H^2 e^{2ikx_j} \frac{i\sin\left(k\Delta x\right)}{\Delta x} He^{ikx_j} \frac{i\sin\left(k\Delta x\right)}{\Delta x} Ue^{ikx_j}  + \frac{1}{3} H^3 e^{3ikx_j} \left(\frac{2\sin\left(\frac{k\Delta x}{2}\right)}{\Delta x}\right)^2 Ue^{ikx_j}\]

\[G_j = UH e^{2ikx_j} + UH^3 e^{4ikx_j} \left(\frac{\sin\left(k\Delta x\right)}{\Delta x}\right)^2  + \frac{1}{3} UH^3 e^{4ikx_j} \left(\frac{2\sin\left(\frac{k\Delta x}{2}\right)}{\Delta x}\right)^2\]

\[G_j = \left(1 + H^2 e^{2ikx_j} \left(\frac{\sin\left(k\Delta x\right)}{\Delta x}\right)^2  + \frac{1}{3} H^2 e^{2ikx_j} \left(\frac{2\sin\left(\frac{k\Delta x}{2}\right)}{\Delta x}\right)^2 \right)UH e^{2ikx_j}\]

\[G_j = \left(1 + \left[\left(\frac{\sin\left(k\Delta x\right)}{\Delta x}\right)^2  + \frac{1}{3} \left(\frac{2\sin\left(\frac{k\Delta x}{2}\right)}{\Delta x}\right)^2\right]H^2 e^{2ikx_j} \right)UH e^{2ikx_j}\]

So we want

\[\left[\left(\frac{\sin\left(k\Delta x\right)}{\Delta x}\right)^2  + \frac{1}{3} \left(\frac{2\sin\left(\frac{k\Delta x}{2}\right)}{\Delta x}\right)^2\right] \approx \frac{4}{3}k^2 \] 

Wolfram Alpha has

\[\left[\left(\frac{\sin\left(k\Delta x\right)}{\Delta x}\right)^2  + \frac{1}{3} \left(\frac{2\sin\left(\frac{k\Delta x}{2}\right)}{\Delta x}\right)^2\right] = \frac{4k^2}{3} - \frac{13 k^4 \Delta x^2}{36} + \frac{49 k^6 \Delta x^4}{1080} + O(\Delta x^6)\]

So we can see this scheme is second order.

\section{Finite Element}

\[G = uh - \frac{\partial}{\partial x}\left(\frac{h^3}{3}u_x\right)\]

To do so we begin by first multiplying by an arbitrary test function $v$ so that

\[Gv = uhv  - \frac{\partial}{\partial x}\left(\frac{h^3}{3}u_x\right)v\]

and then we integrate over the entire domain to get 
\[\int_\Omega Gv dx = \int_\Omega uhv dx - \int_\Omega \frac{\partial}{\partial x}\left(\frac{h^3}{3}u_x\right)vdx\]

for all $v$

We then make use of integration by parts, with Dirchlet boundaries to get

\[\int_\Omega Gv dx = \int_\Omega uhv dx + \int_\Omega\frac{h^3}{3}u_{x}v_xdx\]

For $u$ we are going to use $x_{j + 1/2}$ as the nodes, which generate the basis functions $\phi_{j + 1/2}$, which for us will be the space of continuous linear elements. While for $G$  and $h$ we will choose basis functions $w$ that are linear from $[x_{j-1/2}, x_{j+1/2}]$ but discontinuous at the edges.

\[\sum_{j}\int_{x_{j-1/2}}^{x_{j+3/2}} Gv dx = \sum_{j}\int_{x_{j-1/2}}^{x_{j+3/2}}  uhv dx + \sum_{j}\int_{x_{j-1/2}}^{x_{j+3/2}} \frac{h^3}{3}u_{x}v_{x}dx\]

for all v

\section{P1 FEM}
We are going to cordinate tranform from x space the interval $[x_{j-1/2},x_{j+1/2} ,x_{j+3/2}]$ to the $\xi$ space interval $[-1,0,1]$. To accomplish this we have the following relation

$$x = \xi\Delta x + x_{j+1/2}$$

Taking the derivatives we see


$dx = d\xi\Delta x$ , $\frac{dx}{d\xi} = \Delta x$ , $\frac{d\xi}{dx} = \frac{1}{\Delta x}$ . \\ \\ We can describe the basis functions in the $\xi$ space

\begin{equation}
\phi_{j+1/2} = \left\lbrace \begin{array}{c c}
1 + \xi & \xi < 0 \\
1 - \xi & \xi > 0\\
0 & \text{otherwise}
\end{array} 
\right.
\end{equation}

\begin{equation}
\phi_{j-1/2} = \left\lbrace \begin{array}{c c}
-\xi & \xi < 0 \\
0 & \text{otherwise}
\end{array} 
\right.
\end{equation}

\begin{equation}
\phi_{j+3/2} = \left\lbrace \begin{array}{c c}
\xi & \xi > 0 \\
0 & \text{otherwise}
\end{array} 
\right.
\end{equation}

While the descriptions for $w$'s is
\begin{equation}
w^+_{j+1/2} = \left\lbrace \begin{array}{c c}
1-\xi & \xi > 0 \\
0 & \text{otherwise}
\end{array} 
\right.
\end{equation}

\begin{equation}
w^-_{j+1/2} = \left\lbrace \begin{array}{c c}
1+\xi & \xi < 0 \\
0 & \text{otherwise}
\end{array} 
\right.
\end{equation}

\begin{equation}
w^+_{j-1/2} = \left\lbrace \begin{array}{c c}
-\xi & \xi < 0 \\
0 & \text{otherwise}
\end{array} 
\right.
\end{equation}

\begin{equation}
w^-_{j+3/2} = \left\lbrace \begin{array}{c c}
\xi & \xi > 0 \\
0 & \text{otherwise}
\end{array} 
\right.
\end{equation}

We now replace our functions by our approximations to them

\[G \approx G' = \sum_{j}G_{j+1/2}w_{j+1/2}\]
\[u \approx u' = \sum_{j}u_{j+1/2}\phi_{j+1/2}\]
\[h \approx h' = \sum_{j}h_{j+1/2}w_{j+1/2}\]

\[\sum_{j}\int_{x_{j-1/2}}^{x_{j+3/2}} G'\phi_{j+1/2} dx - \int_{x_{j-1/2}}^{x_{j+3/2}} u'h'\phi_{j+1/2} dx -  \int_{x_{j-1/2}}^{x_{j+3/2}}\frac{(h')^3}{3}u'_{x}(\phi_{x})_{j+1/2}dx = 0 \]

For all $\phi_{j+1/2}$. For this analysis we choose a particular basis function $\phi_{j+1/2}$ and we look at all the integrals. Begining from the right

\[\int_{x_{j-1/2}}^{x_{j+3/2}} G'(x)\phi_{j+1/2} dx = \int_{-1}^{1} G'(\xi)\phi_{j+1/2}(\xi) \frac{d x}{d\xi}d\xi\]

\[= \Delta x \int_{-1}^{1} \left(G^+_{j- 1/2}w^+_{j - 1/2} + G^-_{j+ 1/2}w^-_{j + 1/2} + G^+_{j+ 1/2}w^+_{j + 1/2} + G^-_{j- 3/2}w^-_{j - 3/2}     \right)\phi_{j+1/2} d\xi\]

\begin{multline}
= \Delta x [G^+_{j- 1/2} \int_{-1}^{1} w^+_{j - 1/2}\phi_{j+1/2} d\xi + G^-_{j+ 1/2} \int_{-1}^{1} w^-_{j + 1/2}\phi_{j+1/2} d\xi  \\+ G^+_{j+ 1/2} \int_{-1}^{1} w^+_{j + 1/2}\phi_{j+1/2} d\xi + G^-_{j+ 3/2} \int_{-1}^{1} w^-_{j + 3/2}\phi_{j+1/2} d\xi ]
\end{multline}


We have that

\[\int_{-1}^{1} w^+_{j - 1/2}\phi_{j+1/2} d\xi = \int_{-1}^{1} w^-_{j + 3/2}\phi_{j+1/2} d\xi  = \frac{1}{6}\]

and 

\[\int_{-1}^{1} w^-_{j + 1/2}\phi_{j+1/2} d\xi = \int_{-1}^{1} w^+_{j + 1/2}\phi_{j+1/2} d\xi = \frac{1}{3} \]

So

\[= \Delta x \left[\frac{1}{6}G^+_{j- 1/2} +  \frac{1}{3}G^-_{j+ 1/2} +  \frac{1}{3}G^+_{j+ 1/2}  +  \frac{1}{6}G^-_{j+ 3/2}\right]\]

\[= \frac{\Delta x}{6}  \left[G^+_{j- 1/2} +  2G^-_{j+ 1/2} +  2G^+_{j+ 1/2}  +  G^-_{j+ 3/2}\right]\]

Next we have 
%this chnage of variables fine?

\[\int_{x_{j-1/2}}^{x_{j+3/2}} h'u'\phi_{j+1/2} dx = \Delta x \int_{-1}^{1} h'(\xi)u'(\xi)\phi_{j+1/2}(\xi) d\xi\]
\begin{multline}
= \Delta x \int_{-1}^{1} \left(h^+_{j- 1/2}w^+_{j - 1/2} + h^-_{j+ 1/2}w^-_{j + 1/2} + h^+_{j+ 1/2}w^+_{j + 1/2} + h^-_{j- 3/2}w^-_{j - 3/2}\right)\\\left(u_{j- 1/2}\phi_{j - 1/2} + u_{j+1/2}\phi_{j+1/2} +u_{j+ 3/2}\phi_{j+ 3/2} \right)\phi_{j+1/2} d\xi
\end{multline}

\begin{multline}
= \Delta x \int_{-1}^{1} \left(h^+_{j- 1/2}w^+_{j - 1/2} + h^-_{j+ 1/2}w^-_{j + 1/2} + h^+_{j+ 1/2}w^+_{j + 1/2} + h^-_{j- 3/2}w^-_{j - 3/2}\right)u_{j- 1/2}\phi_{j - 1/2}\phi_{j+1/2} \\
+ \left(h^+_{j- 1/2}w^+_{j - 1/2} + h^-_{j+ 1/2}w^-_{j + 1/2} + h^+_{j+ 1/2}w^+_{j + 1/2} + h^-_{j- 3/2}w^-_{j - 3/2}\right)u_{j+1/2}\phi_{j+1/2}\phi_{j+1/2} \\
+\left(h^+_{j- 1/2}w^+_{j - 1/2} + h^-_{j+ 1/2}w^-_{j + 1/2} + h^+_{j+ 1/2}w^+_{j + 1/2} + h^-_{j- 3/2}w^-_{j - 3/2}\right)u_{j+ 3/2}\phi_{j+ 3/2}\phi_{j+1/2} d\xi
\end{multline}

If one of the terms $w_k$, $\phi_l$ , $\phi_m$ is 0 then  $w_k\phi_l\phi_m = 0 $

\begin{multline}
= \Delta x \int_{-1}^{1} \left(h^+_{j- 1/2}w^+_{j - 1/2} + h^-_{j+ 1/2}w^-_{j + 1/2} \right)u_{j- 1/2}\phi_{j - 1/2}\phi_{j+1/2} \\
+ \left(h^+_{j- 1/2}w^+_{j - 1/2} + h^-_{j+ 1/2}w^-_{j + 1/2} + h^+_{j+ 1/2}w^+_{j + 1/2} + h^-_{j- 3/2}w^-_{j - 3/2}\right)u_{j+1/2}\phi_{j+1/2}\phi_{j+1/2} \\
+\left(h^+_{j+ 1/2}w^+_{j + 1/2} + h^-_{j- 3/2}w^-_{j - 3/2}\right)u_{j+ 3/2}\phi_{j+ 3/2}\phi_{j+1/2} d\xi
\end{multline}

\begin{multline}
= \Delta x \int_{-1}^{1} u_{j- 1/2}h^+_{j- 1/2}w^+_{j - 1/2}\phi_{j - 1/2}\phi_{j+1/2} + u_{j- 1/2}h^-_{j+ 1/2}w^-_{j + 1/2}\phi_{j - 1/2}\phi_{j+1/2} \\
+  u_{j+1/2}h^+_{j- 1/2}w^+_{j - 1/2}\phi_{j+1/2}\phi_{j+1/2}  + u_{j+1/2}h^-_{j+ 1/2}w^-_{j + 1/2}\phi_{j+1/2}\phi_{j+1/2}\\ + u_{j+1/2} h^+_{j+ 1/2}w^+_{j + 1/2}\phi_{j+1/2}\phi_{j+1/2}  + u_{j+1/2} h^-_{j- 3/2}w^-_{j - 3/2}\phi_{j+1/2}\phi_{j+1/2} \\
+u_{j+ 3/2}h^+_{j+ 1/2}w^+_{j + 1/2}\phi_{j+ 3/2}\phi_{j+1/2} + u_{j+ 3/2}h^-_{j- 3/2}w^-_{j - 3/2}\phi_{j+ 3/2}\phi_{j+1/2} d\xi
\end{multline}

Evaluating the integral

\begin{multline}
\int_{-1}^{1} u_{j- 1/2}h^+_{j- 1/2}w^+_{j - 1/2}\phi_{j - 1/2}\phi_{j+1/2} + u_{j- 1/2}h^-_{j+ 1/2}w^-_{j + 1/2}\phi_{j - 1/2}\phi_{j+1/2} \\
+  u_{j+1/2}h^+_{j- 1/2}w^+_{j - 1/2}\phi_{j+1/2}\phi_{j+1/2}  + u_{j+1/2}h^-_{j+ 1/2}w^-_{j + 1/2}\phi_{j+1/2}\phi_{j+1/2}\\ + u_{j+1/2} h^+_{j+ 1/2}w^+_{j + 1/2}\phi_{j+1/2}\phi_{j+1/2}  + u_{j+1/2} h^-_{j- 3/2}w^-_{j - 3/2}\phi_{j+1/2}\phi_{j+1/2} \\
+u_{j+ 3/2}h^+_{j+ 1/2}w^+_{j + 1/2}\phi_{j+ 3/2}\phi_{j+1/2} + u_{j+ 3/2}h^-_{j- 3/2}w^-_{j - 3/2}\phi_{j+ 3/2}\phi_{j+1/2} d\xi
\end{multline}


\begin{multline}
=u_{j- 1/2}h^+_{j- 1/2}\int_{-1}^{1} w^+_{j - 1/2}\phi_{j - 1/2}\phi_{j+1/2}d\xi  + u_{j- 1/2}h^-_{j+ 1/2}\int_{-1}^{1}w^-_{j + 1/2}\phi_{j - 1/2}\phi_{j+1/2} d\xi\\
+  u_{j+1/2}h^+_{j- 1/2}\int_{-1}^{1} w^+_{j - 1/2}\phi_{j+1/2}\phi_{j+1/2} d\xi  + u_{j+1/2}h^-_{j+ 1/2}\int_{-1}^{1} w^-_{j + 1/2}\phi_{j+1/2}\phi_{j+1/2} d\xi \\ + u_{j+1/2} h^+_{j+ 1/2}\int_{-1}^{1} w^+_{j + 1/2}\phi_{j+1/2}\phi_{j+1/2} d\xi   + u_{j+1/2} h^-_{j- 3/2}\int_{-1}^{1} w^-_{j - 3/2}\phi_{j+1/2}\phi_{j+1/2} d\xi \\
+u_{j+ 3/2}h^+_{j+ 1/2}\int_{-1}^{1} w^+_{j + 1/2}\phi_{j+ 3/2}\phi_{j+1/2} d\xi + u_{j+ 3/2}h^-_{j- 3/2} \int_{-1}^{1} w^-_{j - 3/2}\phi_{j+ 3/2}\phi_{j+1/2} d\xi
\end{multline}

Now we evaluate the integrals

\[\int_{-1}^{1} w^+_{j - 1/2}\phi_{j - 1/2}\phi_{j+1/2}d\xi = \int_{-1}^{0}\left(-\xi\right)\left(-\xi\right)\left(1 + \xi\right)  =  \frac{1}{12 }\]

\[\int_{-1}^{1}w^-_{j + 1/2}\phi_{j - 1/2}\phi_{j+1/2} d\xi =  \int_{-1}^{0}\left(1 + \xi\right)\left(-\xi\right) \left(1 + \xi\right) d\xi = \frac{1}{12}\]

\[\int_{-1}^{1} w^+_{j - 1/2}\phi_{j+1/2}\phi_{j+1/2} d\xi =  \int_{-1}^{0}\left(-\xi\right)\left(1 + \xi\right) \left(1 + \xi\right) d\xi= \frac{1}{12} \]

\[\int_{-1}^{1} w^-_{j + 1/2}\phi_{j+1/2}\phi_{j+1/2} d\xi = \int_{-1}^{0}\left(1 + \xi\right)\left(1 + \xi\right) \left(1 + \xi\right) d\xi = \frac{1}{4}\]
\[\int_{-1}^{1} w^+_{j + 1/2}\phi_{j+1/2}\phi_{j+1/2} d\xi = \int_{0}^{1}\left(1 - \xi\right)\left(1 - \xi\right)\left(1 - \xi\right) d\xi= \frac{1}{4} \]

\[\int_{-1}^{1} w^-_{j - 3/2}\phi_{j+1/2}\phi_{j+1/2} d\xi =\int_{0}^{1}\left(\xi\right)\left(1 - \xi\right)\left(1 - \xi\right) d\xi = \frac{1}{12} \]

\[\int_{-1}^{1} w^+_{j + 1/2}\phi_{j+ 3/2}\phi_{j+1/2} d\xi = \int_{0}^{1}\left(1 - \xi\right)\left(\xi\right)\left(1 - \xi\right) d\xi = \frac{1}{12} \]

\[\int_{-1}^{1} w^-_{j - 3/2}\phi_{j+ 3/2}\phi_{j+1/2} d\xi = \int_{0}^{1}\left(\xi\right)\left(\xi\right)\left(1 - \xi\right) d\xi = \frac{1}{12}  \]

Note that these sum to the same fractions as in the linear case if the h is constant. 

\begin{multline}
=u_{j- 1/2}h^+_{j- 1/2}\frac{1}{12 } + u_{j- 1/2}h^-_{j+ 1/2}\frac{1}{12 }\\
+  u_{j+1/2}h^+_{j- 1/2}\frac{1}{12 } + u_{j+1/2}h^-_{j+ 1/2}\frac{1}{4} \\ + u_{j+1/2} h^+_{j+ 1/2}\frac{1}{4 }   + u_{j+1/2} h^-_{j- 3/2}\frac{1}{12 } \\
+u_{j+ 3/2}h^+_{j+ 1/2}\frac{1}{12 } + u_{j+ 3/2}h^-_{j- 3/2}\frac{1}{12 }
\end{multline}

\begin{multline}
= \frac{1}{12} \bigg[ u_{j- 1/2}h^+_{j- 1/2} + u_{j- 1/2}h^-_{j+ 1/2}
+  u_{j+1/2}h^+_{j- 1/2} + 3u_{j+1/2}h^-_{j+ 1/2} \\ + 3u_{j+1/2} h^+_{j+ 1/2}  + u_{j+1/2} h^-_{j- 3/2}
+u_{j+ 3/2}h^+_{j+ 1/2} + u_{j+ 3/2}h^-_{j- 3/2} \bigg]
\end{multline}

Therefore
\begin{multline}
\int_{x_{j-1/2}}^{x_{j+3/2}} h'u'\phi_{j+1/2} dx= \frac{\Delta x}{12} \bigg[ u_{j- 1/2}h^+_{j- 1/2} + u_{j- 1/2}h^-_{j+ 1/2}
+  u_{j+1/2}h^+_{j- 1/2} + 3u_{j+1/2}h^-_{j+ 1/2} \\ + 3u_{j+1/2} h^+_{j+ 1/2}  + u_{j+1/2} h^-_{j- 3/2}
+u_{j+ 3/2}h^+_{j+ 1/2} + u_{j+ 3/2}h^-_{j- 3/2} \bigg]
\end{multline}

The next integral is 

\[\int_{x_{j-1/2}}^{x_{j+3/2}}\frac{(h')^3}{3}u'_{x}(\phi_{x})_{j+1/2}dx = \Delta x \int_{-1}^{1} \frac{(h'(\xi))^3}{3}u'_{\xi}(\phi_{\xi})_{j+1/2}dx\]

\[ = \frac{\Delta x}{3} \int_{-1}^{1} (h'(\xi))^3u'_{\xi}(\phi_{\xi})_{j+1/2}d\xi\]

were now going to expand andd use the superscript $'$ to denote derivatives

\begin{multline}
= \frac{\Delta x}{3} \int_{-1}^{1} \left(h^+_{j- 1/2}w^+_{j - 1/2} + h^-_{j+ 1/2}w^-_{j + 1/2} + h^+_{j+ 1/2}w^+_{j + 1/2} + h^-_{j- 3/2}w^-_{j - 3/2}\right)^3 \\ \left(u_{j- 1/2}\phi'_{j - 1/2} + u_{j+1/2}\phi'_{j+1/2} +u_{j+ 3/2}\phi'_{j+ 3/2} \right)\phi_{j+1/2}'d\xi
\end{multline}





\end{document}
