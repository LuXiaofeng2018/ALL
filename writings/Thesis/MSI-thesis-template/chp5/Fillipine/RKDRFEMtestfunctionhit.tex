\documentclass[12pt]{article}
\usepackage{amsmath}
\usepackage{ amssymb }
\usepackage{breqn}
\begin{document}

\section{Elliptic Equation}
The linearised elliptic equation is

\[G = Hu - \frac{H^3}{3}u_{xx}\]

Want to find out the FEM approximation to this $G'$ such that

\[G' = \mathcal{G}_{FE_1} u\]

for $P^1$ FEM

\[G' = \mathcal{G}_{FE_2} u\]

for $P^2$ FEM.

To do so we begin by first multiplying by an arbitrary test function $v$ so that

\[Gv = Huv - \frac{H^3}{3}u_{xx}v\]

and then we integrate over the entire domain to get 
\[\int_\Omega Gv dx = \int_\Omega Huv dx - \int_\Omega \frac{H^3}{3}u_{xx}vdx\]

for all $v$

We then make use of integration by parts, with Dirchlet boundaries to get

\[\int_\Omega Gv dx = \int_\Omega Huv dx + \int_\Omega \frac{H^3}{3}u_{x}v_xdx\]

Our FVM discretisation already has a natrual structure with linear functions intervals of $[x_{j- 1/2} , x_{j+1/2}]$, to achieve this in $P^1$ we have our nodes at the boundaries, thus

So we can reformulate this as 

\[\sum_{j}\int_{x_{j-1/2}}^{x_{j+3/2}} Gv dx = \sum_{j}\int_{x_{j-1/2}}^{x_{j+3/2}} Huv dx + \sum_{j}\int_{x_{j-1/2}}^{x_{j+3/2}} \frac{H^3}{3}u_{x}v_{x}dx\]

or more aptly

\[\sum_{j}\int_{x_{j-1/2}}^{x_{j+3/2}} Gv dx - \int_{x_{j-1/2}}^{x_{j+3/2}} Huv dx - \int_{x_{j-1/2}}^{x_{j+3/2}} \frac{H^3}{3}u_{x}v_{x}dx = 0 \]

for all v

\[\sum_{j}\int_{x_{j-1/2}}^{x_{j+3/2}} Gv dx - H\int_{x_{j-1/2}}^{x_{j+3/2}} uv dx -  \frac{H^3}{3}\int_{x_{j-1/2}}^{x_{j+3/2}}u_{x}v_{x}dx = 0 \]

\section{P1 FEM}
We have the basis functions for linear elements:

\[v_{j + 1/2} = \left\lbrace \begin{array}{c c}
\frac{x - x_{j-1/2}}{\Delta x}  & x_{j - 1/2} \le x < x_{j+1/2} \\
1 - \frac{x - x_{j+1/2}}{\Delta x}  & x_{j + 1/2} \le x < x_{j+3/2}\\
0 & \text{otherwise}
\end{array} \right.\]
Thus

\[(v_{j + 1/2})_{x} = \left\lbrace \begin{array}{c c}
\frac{1}{\Delta x}  & x_{j - 1/2} \le x < x_{j+1/2} \\
- \frac{1}{\Delta x}  & x_{j + 1/2} \le x < x_{j+3/2}\\
0 & \text{otherwise}
\end{array} \right.\]

For this FEM we are intereseted in $G_{i+1/2}$ and then we can just get a shift operator to get the otherones. For FEM we replace the functions by their P1 approximations so

\[G \approx G' = \sum^{j}G_{j+ 1/2}v_{j+ 1/2}\]
\[u \approx u' = \sum^{j}u_{j+ 1/2}v_{j+ 1/2}\]

We hit our weak formulation with the test function $v_{j + 1/2}$ with our P1 approximations

\[\sum_{j}\int_{x_{j-1/2}}^{x_{j+3/2}} G'v_{j+ 1/2} dx - H\int_{x_{j-1/2}}^{x_{j+3/2}} u'v_{j+ 1/2} dx -  \frac{H^3}{3}\int_{x_{j-1/2}}^{x_{j+3/2}}u'_{x}(v_{j+ 1/2})_{x}dx = 0 \]

This test function is zero on all elements except $j$ thus

\begin{multline}
\int_{x_{j-1/2}}^{x_{j+3/2}} \left(G_{j- 1/2}v_{j- 1/2} + G_{j+ 1/2}v_{j+ 1/2} +G_{j+3/2}v_{j+3/2}   \right)v_{j+ 1/2} dx \\ - H\int_{x_{j-1/2}}^{x_{j+3/2}} \left(u_{j- 1/2}v_{j- 1/2} + u_{j+ 1/2}v_{j+ 1/2} +u_{j+3/2}v_{j+3/2}   \right)v_{j+ 1/2} dx - \\ \frac{H^3}{3}\int_{x_{j-1/2}}^{x_{j+3/2}}\left(u_{j- 1/2}(v_{j- 1/2})_{x} + u_{j+ 1/2}(v_{j+ 1/2})_{x} +u_{j+3/2}(v_{j+ 3 /2})_{x}   \right)(v_{j+ 1/2})_{x}dx = 0
\end{multline}

To calculate these we just need expressions for 

\[\int_{x_{j-1/2}}^{x_{j+3/2}}v_{j- 1/2}v_{j+ 1/2} dx\]

\[\int_{x_{j-1/2}}^{x_{j+3/2}}v_{j+ 1/2}v_{j+ 1/2} dx\]

\[\int_{x_{j-1/2}}^{x_{j+3/2}}v_{j+3/2}v_{j+ 1/2} dx\]

and 

\[\int_{x_{j-1/2}}^{x_{j+3/2}}(v_{j- 1/2})_x(v_{j+ 1/2})_x dx\]

\[\int_{x_{j-1/2}}^{x_{j+3/2}}(v_{j+ 1/2})_x(v_{j+ 1/2})_x dx\]

\[\int_{x_{j-1/2}}^{x_{j+3/2}}(v_{j+ 3/2})_x(v_{j+ 1/2})_x dx\]


\[v_{j + 1/2} = \left\lbrace \begin{array}{c c}
\frac{x - x_{j-1/2}}{\Delta x}  & x_{j - 1/2} \le x < x_{j+1/2} \\
1 - \frac{x - x_{j+1/2}}{\Delta x}  & x_{j + 1/2} \le x < x_{j+3/2}\\
0 & \text{otherwise}
\end{array} \right.\]
Thus

\[(v_{j + 1/2})_{x} = \left\lbrace \begin{array}{c c}
\frac{1}{\Delta x}  & x_{j - 1/2} \le x < x_{j+1/2} \\
- \frac{1}{\Delta x}  & x_{j + 1/2} \le x < x_{j+3/2}\\
0 & \text{otherwise}
\end{array} \right.\]

We begin:

\[\int_{x_{j-1/2}}^{x_{j+3/2}}v_{j- 1/2}v_{j+ 1/2} dx = \int_{x_{j-1/2}}^{x_{j+1/2}}v_{j- 1/2}v_{j+ 1/2} dx \]
\[ = \int_{x_{j-1/2}}^{x_{j+1/2}}(1 - \frac{x - x_{j-1/2}}{\Delta x})\frac{x - x_{j-1/2}}{\Delta x} dx \]

\[ = \int_{x_{j-1/2}}^{x_{j+1/2}}\frac{x - x_{j-1/2}}{\Delta x} - \left(\frac{x - x_{j-1/2}}{\Delta x}\right)^2 dx \]

\[ = \int_{x_{j-1/2}}^{x_{j+1/2}}\frac{x - x_{j-1/2}}{\Delta x} - \left(\frac{x^2 - 2x x_{j-1/2} + x^2_{j-1/2}}{\Delta x^2}\right) dx \]

\[ = \int_{x_{j-1/2}}^{x_{j+1/2}}\frac{\Delta x x - \Delta x x_{j-1/2}}{\Delta x^2} - \left(\frac{x^2 - 2x x_{j-1/2} + x^2_{j-1/2}}{\Delta x^2}\right)dx \]

\[ = \int_{x_{j-1/2}}^{x_{j+1/2}}\frac{\Delta x x - \Delta x x_{j-1/2}}{\Delta x^2} - \frac{x^2 - 2x x_{j-1/2} + x^2_{j-1/2}}{\Delta x^2} dx \]

\[ = \int_{x_{j-1/2}}^{x_{j+1/2}}\frac{\Delta x x - \Delta x x_{j-1/2} - x^2 + 2x x_{j-1/2} - x^2_{j-1/2}}{\Delta x^2} dx \]

\[ = \int_{x_{j-1/2}}^{x_{j+1/2}}\frac{- x^2  +(\Delta x + 2x_{j-1/2} ) x - \Delta x x_{j-1/2} - x^2_{j-1/2}}{\Delta x^2} dx \]

\[ = \frac{1}{\Delta x^2}\int_{x_{j-1/2}}^{x_{j+1/2}}- x^2  +(\Delta x + 2x_{j-1/2} ) x - \Delta x x_{j-1/2} - x^2_{j-1/2} dx \]

\[ = \frac{1}{\Delta x^2} \left[- \frac{1}{3}x^3  + \frac{1}{2}(\Delta x + 2x_{j-1/2} ) x^2 + (-\Delta x x_{j-1/2} - x^2_{j-1/2})x  \right]_{x_{j-1/2}}^{x_{j+1/2}} \]

\begin{multline}
\frac{1}{\Delta x^2} \left[- \frac{1}{3}x_{j+1/2}^3  + \frac{1}{2}(\Delta x + 2x_{j-1/2} ) x_{j+1/2}^2 + (-\Delta x x_{j-1/2} - x^2_{j-1/2})x_{j+1/2}  \right]\\ - \frac{1}{\Delta x^2} \left[- \frac{1}{3}x_{j-1/2}^3  + \frac{1}{2}(\Delta x + 2x_{j-1/2} ) x_{j-1/2}^2 + (-\Delta x x_{j-1/2} - x^2_{j-1/2})x_{j-1/2} \right]
\end{multline}

\end{document}
